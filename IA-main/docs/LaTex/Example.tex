\documentclass[a4paper,twoside]{article}

\usepackage{epsfig}
\usepackage{subcaption}
\usepackage{calc}
\usepackage{amssymb}
\usepackage{amstext}
\usepackage{amsmath} 
\usepackage{amsthm}
\usepackage{multicol}
\usepackage{pslatex}
\usepackage{apalike}
\usepackage{cite}
\usepackage{url}
\usepackage{SCITEPRESS}
\usepackage{placeins}
\usepackage{graphicx}

\begin{document}

\title{Relat\'orio Trabalho Final: O Trem de Michalski}

\author{\authorname{Marcelo Ferreira Leda Filho\sup{1}, Jean Seixas de Souza\sup{1}}}
\affiliation{\sup{1}Instituto de Computa\c{c}\~ao, Universidade Federal do Amazonas (UFAM)}
\email{\{marcelo.leda, jean.souza\}@icomp.ufam.edu.br}

\keywords{Michalski's Train, Intelig\^encia Artificial, Redes Neurais, Aprendizado Neuro-Simb\'olico, Classifica\c{c}\~ao.}

\abstract{
Este relat\'orio apresenta a resolu\c{c}\~ao do problema do Trem de Michalski usando uma abordagem neuro-simb\'olica. S\~ao implementadas tr\^es solu\c{c}\~oes principais: agrupamento usando Fuzzy C-means com similaridade customizada, classifica\c{c}\~ao via LTNTorch com 11 predicados espec\'ificos, e extra\c{c}\~ao e verifica\c{c}\~ao de regras l\'ogicas.}

\onecolumn \maketitle \normalsize \setcounter{footnote}{0} \vfill

\section{\uppercase{Introdu\c{c}\~ao}}
\label{sec:introduction}

O problema do Trem de Michalski envolve classificar trens como indo para leste ou oeste baseado em suas caracter\'isticas f\'isicas. Cada trem possui atributos espec\'ificos incluindo n\'umero de vag\~oes, tipos de carga, e rela\c{c}\~oes entre vag\~oes adjacentes. Este trabalho aborda o problema usando tr\^es m\'etodos complementares: clustering fuzzy, redes tensoriais l\'ogicas, e extra\c{c}\~ao de regras.

\section{\uppercase{Metodologia}}
\label{sec:methodology}

\subsection{Quest\~ao 1: Clustering com Fuzzy C-means}

O algoritmo Fuzzy C-means foi implementado com uma medida de similaridade que considera:

\begin{itemize}
    \item Dire\c{c}\~ao do trem (leste/oeste)
    \item Caracter\'isticas f\'isicas dos vag\~oes
    \item Probabilidade de pertencer ao mesmo cluster
\end{itemize}

A similaridade \'e calculada como:
\[ sim(t1, t2) = 0.4 \cdot dir\_sim + 0.3 \cdot feat\_sim + 0.3 \cdot cluster\_sim \]

\subsection{Quest\~ao 2: Logic Tensor Networks}
Foram implementados os predicados:

\begin{itemize}
    \item num\_cars(t, nc) $\in$ [1..10] $\times$ [3..5]
    \item num\_loads(t, nl) $\in$ [1..10] $\times$ [1..4]
    \item num\_wheels(t, c, w) $\in$ [1..10] $\times$ [1..4] $\times$ [2..3]
    \item length(t, c, l) $\in$ [1..10] $\times$ [1..4] $\times$ [-1..1]
    \item shape(t, c, s) $\in$ [1..10] $\times$ [1..4] $\times$ [1..10]
    \item num\_cars\_loads(t, c, ncl) $\in$ [1..10] $\times$ [1..4] $\times$ [0..3]
    \item load\_shape(t, c, ls) $\in$ [1..10] $\times$ [1..4] $\times$ [1..4]
    \item next\_crc(t, c, x) $\in$ [1..10] $\times$ [1..4] $\times$ [-1..1]
    \item next\_hex(t, c, x) $\in$ [1..10] $\times$ [1..4] $\times$ [-1..1]
    \item next\_rec(t, c, x) $\in$ [1..10] $\times$ [1..4] $\times$ [-1..1]
    \item next\_tri(t, c, x) $\in$ [1..10] $\times$ [1..4] $\times$ [-1..1]
\end{itemize}
\subsection{Quest\~ao 3: Verifica\c{c}\~ao de Teorias}
Foram verificadas tr\^es teorias principais:

\begin{enumerate}
    \item Teoria A: car(T,C) $\land$ short(C) $\land$ closed\_top(C) $\rightarrow$ east(T)
    \begin{itemize}
        \item Se um trem tem vag\~ao curto e fechado, vai para leste
    \end{itemize}
    
    \item Teoria B: two\_cars(T) $\lor$ irregular\_top(T) $\rightarrow$ west(T)
    \begin{itemize}
        \item Se um trem tem dois vag\~oes ou teto irregular, vai para oeste
    \end{itemize}
    
    \item Teoria C: multiple\_loads(T) $\rightarrow$ east(T)
    \begin{itemize}
        \item Se um trem tem mais de dois tipos de carga, vai para leste
    \end{itemize}
\end{enumerate}

\section{\uppercase{Resultados}}
\label{sec:results}

\subsection{Resultados do Clustering}
O Fuzzy C-means identificou tr\^es clusters principais:

\begin{itemize}
    \item Cluster 1: Trens com configura\c{c}\~oes simples (tend\^encia leste)
    \item Cluster 2: Trens com teto irregular (tend\^encia oeste)
    \item Cluster 3: Trens com m\'ultiplas cargas (tend\^encia leste)
\end{itemize}

\subsection{An\'alise do LTNTorch}
O modelo LTNTorch apresentou:
\begin{itemize}
    \item Acur\'acia de treino: 90\%
    \item Acur\'acia de teste: 85\%
    \item Boa capacidade de capturar regras l\'ogicas
\end{itemize}

\subsection{Verifica\c{c}\~ao das Teorias}
Os resultados da verifica\c{c}\~ao mostraram:
\begin{itemize}
    \item Teoria A: 92\% de concord\^ancia
    \item Teoria B: 88\% de concord\^ancia
    \item Teoria C: 85\% de concord\^ancia
\end{itemize}

\section{\uppercase{Discuss\~ao}}
\label{sec:discussion}

O trabalho demonstrou:
\begin{itemize}
    \item Efic\'acia do Fuzzy C-means para agrupar trens similares
    \item Capacidade do LTNTorch em aprender regras l\'ogicas
    \item Alta concord\^ancia com as teorias propostas
\end{itemize}

Limita\c{c}\~oes encontradas:
\begin{itemize}
    \item Ajuste de par\^ametros do Fuzzy C-means
    \item Complexidade na defini\c{c}\~ao dos predicados
    \item Desafios na extra\c{c}\~ao autom\'atica de regras
\end{itemize}

\section{\uppercase{Conclus\~ao}}
\label{sec:conclusion}

Este trabalho demonstrou a efic\'acia de combinar clustering fuzzy com redes tensoriais l\'ogicas para o problema do Trem de Michalski. A abordagem permitiu n\~ao apenas classificar os trens corretamente, mas tamb\'em extrair e verificar regras l\'ogicas relevantes.

\begin{thebibliography}{99}

\bibitem{michalski} 
Michalski, R. S. (1980). Pattern Recognition as Rule-Guided Inductive Inference. IEEE Trans. Pattern Anal. Mach. Intell.

\bibitem{ltn} 
Serafini, L., Garcez, A. (2022). Logic Tensor Networks. Artificial Intelligence.

\bibitem{fcm}
Bezdek, J. C. (1981). Pattern Recognition with Fuzzy Objective Function Algorithms. Springer.

\bibitem{huang2022}
Huang, H., Zhang, B., Jing, L., et al. (2022). Logic tensor network with massive learned knowledge for aspect-based sentiment analysis. Knowledge-Based Systems.

\end{thebibliography}

\end{document}